\documentclass[a4paper]{article}
\usepackage[T1]{fontenc}
\usepackage[utf8]{inputenc}
\usepackage{natbib}
\usepackage{hyperref}
\usepackage{multirow}
\usepackage{tipa} % IPA
\usepackage{amsmath} % formulae
\usepackage{mathtools} % displaystyle in formulae with cases
\usepackage{forest} % (non-dendrogram) trees
\usetikzlibrary{decorations.pathreplacing} % draw braces in tikz figure
\usepackage{standalone} % import figures
\usepackage{changepage} % adjust text width
\usepackage{parskip} % proper paragraphs, no indentation
% \usepackage{showframe} % TODO remove
\usepackage[margin=3cm]{geometry}

% \linespread{1.25}

\title{Clustering Dialect Varieties Based on Historical Sound Correspondences}
\author{Verena Blaschke}
\date{\today}

\begin{document}

\begin{titlepage}
\begin{center}

\vspace*{.15\textheight}

{\Large Bachelor's Thesis}
\vspace{2em}

\hrule
\vspace{0.6cm}
{\huge\bfseries
Clustering Dialect Varieties Based on Historical Sound Correspondences
}\\[0.7cm] 
\hrule
\vspace*{.05\textheight}
 
\begin{minipage}[t]{0.4\textwidth}
\begin{flushleft} 
{\large
\textit{Author}\\
Verena Blaschke}\\
\href{mailto:verena.blaschke@student.uni-tuebingen.de}{\textit{verena.blaschke@student.uni-tuebingen.de}}\\
\end{flushleft}
\end{minipage}
\begin{minipage}[t]{0.4\textwidth}
\begin{flushright}
{\large
\textit{Supervisor}\\
Dr. Çağrı Çöltekin}\\
\href{mailto:ccoltekin@sfs.uni-tuebingen.de}{\textit{ccoltekin@sfs.uni-tuebingen.de}}\\
\end{flushright}
\end{minipage}\\

\vfill

A thesis submitted in partial fulfillment\\
of the requirements for the degree of\\[2mm]
{\large Bachelor of Arts}\\
in\\[1mm]
{\large International Studies in Computational Linguistics}

\vspace*{.1\textheight}

{\large Seminar für Sprachwissenschaft\\
Eberhard Karls Universität Tübingen

\vspace{1em}
August 2018}
\end{center}
  
\end{titlepage}




\pagenumbering{gobble}
% TODO abstract
\newpage
\tableofcontents
\listoftables
\listoffigures
\newpage
% TODO Eigenständigkeits- bzw. Antiplagiatserklärung  

\pagenumbering{arabic}


\section{Introduction}

% --- on clustering dialects and determining relevant features ---

clustering dialects and determining relevant features

\citet{prokic2012detecting}


\citet{heggarty2010splits} created a NeighborNet for Germanic languages, based on pronunciation differences between modern varieties. 
~conclude~ that this works because the pronunciation differences implicitly include information on sound changes that were not shared by all varieties

\citet{prokic2013combining}: judging regularity of correspondences via contingency table

% --- on historical sound correspondences ---

historical sound correspondences

% \citet{kondrak2009identification}
% https://pdfs.semanticscholar.org/d2ea/1dbe1a81f60f99dab04dffc957622b8cb9f2.pdf
% https://www.clips.uantwerpen.be/~gillis/pdf/20040107.9620.cslfinal.pdf
% https://www.aclweb.org/anthology/J/J96/J96-4003.pdf

% --- on BSGC ---

Bipartite spectral graph co-clustering

introduced in \citet{dhillon2001co-clustering}

introduced as a method for dialect clustering in \citet{wieling2011bipartite}

\citet{wieling2013analyzing} applied this method for clustering British English dialects and compared the results to clusters obtained via PCA
 
\citet{montemagni2013synchronic} applied this method to Tuscan dialects and used left and right contexts for the sound segments

\citet{zha2001bipartite} introduced a similar (the same??) method as Dhillon at the same time, in another journal

% These algorithms and related ones were investigated by \citet{kluger2003spectral} for their use in bioinformatics as well (co-clustering genes and phenotypes).

\subsection{Continental West Germanic Doculects}

even West Germanic as its own Germanic branch is contested, though generally accepted (\citet{voyles1971problem}; \citet[pp. 7-8]{harbert2007germanic}; \citet{ringe2012cladistic}) % \citet[p. 404]{krogh1996stellung}

similarities not only because of genetic relatedness but also--enabled by the geographic proximity--mutual influences
cf. \citet[p. 8]{harbert2007germanic}

\citet[pp. 72-80]{nielsen1989germanic} gives an overview of the history of attempting to divide the West Germanic dialects into subgroups with the associated criteria (phonological, morphological, lexical, and/or extra-linguistic) and criticisms.

more recently(?), also interactions between dialects and standard languages \citep{coetsem1992interaction}
\citet{kremer1990einfuehrung}: in Germany and the Netherlands (but not in Switzerland), dialects tend to become closer to the standard languages (state/standard language borders as dialect borders). Low German dialects tend to be replaced however. In Non-Germanic regions (France, Belgium): also replaced

sound changes that characterize some but not all CWG variants:

- High German consonant shift
  - some notes. try \citet[pp. 47-48]{harbert2007germanic} or Beekes
  - \citet[p. 15]{harbert2007germanic}: mostly in Alpine German, partially in Franconian, not in Low German


\begin{figure}
\centering
\includestandalone[width=\textwidth]{figures/harbert}
\caption{Harbert 2007: somewhat taking into account the convergence of certain Germanic varieties due to mutual influence (tree + waves?)}
\label{fig:cwg_harbert}
\end{figure}


\begin{figure}
\begin{adjustwidth}{-3cm}{-3cm}
\centering
\scalebox{0.8}{
\documentclass{standalone}
\usepackage[utf8]{inputenc}
\usepackage{forest}
\begin{document}
\begin{forest}
short/.style={l sep=1mm},
for tree={
  parent anchor=south, 
  child anchor=north,
  align=center, % necessary for intra-node linebreaks
  l sep=3cm, % level
  s sep=0.5mm, % sibling
%   inner sep=0.2mm
}
[West Germanic
  [North Sea Germanic
    [Anglo-Frisian, short
    [Frisian
        [Western\\Frisian\\\textit{Grou}]
        [Northern\\Frisian
            [Ferring\\\textit{Feer}]
            [Helgoland\\\textit{Heligoland}]
        ]
    ]
    ] 
    [Alts\"{a}chsisch\\{[Old Saxon]}, short
    [Middle-Modern\\Low German, short
    [Low German
        [Achter-\\hoeks\\\textit{Achterhoek}]
        [Ost-\\friesisch-\\Groningisch, short  % TODO translation
        [Gronings
            [Veenkolonials\\\textit{Veenkoloni\"{e}n}]
            [Westerwolds\\\textit{Westerkwartier}]
        ]
        ]
    ]
    ]
    ]
  ]
  [Franconian
    [Low Franconian, short
    [Macro-Dutch, short
    [Modern Dutch
        [Dutch\\\textit{Std. Dutch}\\\textit{(NL)}]
        [Vlaams\\\textit{Std. Dutch}\\\textit{(BE)}
            [Ostvlaams\\\textit{Ostend}]
            [Antwerps\\\textit{Antwerp}]
            [Limburgs\\\textit{Limburg}]
        ]
    ]
    ]
    ]
    [High Franconian
        [German, short
        [Alsatian\\\textit{Herrlisheim}]
        ]
        [Middle\\Franconian
            [Luxem-\\bourgish\\\textit{Luxembourg}]
            [Ripurarian, short
            [Kölsch\\\textit{Cologne}]
            ]
        ]
    ]
  ]
    [High German, short
    [Middle-Modern\\High German, short
    [Modern\\High German, short
    [Alpine Germanic
    [Alemannic
        [Swiss\\German
            [Basel\\\textit{Biel}]
            [Low\\Alemannic\\\textit{Hard}]
            [Graubenden-\\Grisons\\\textit{Graub\"{u}nden}]
        ]
        [Swabian\\\textit{T\"{u}bingen}]
        [Walser\\\textit{Walser}]
    ]
    [Bayerisch\\{[Bavarian]}
    [Cimbrian\\\textit{Ortisei}]
    ]
  ]
  ]
  ]
  ]
]
\end{forest}
\end{document}

}
\end{adjustwidth}
% \includestandalone[width=\textwidth]{figures/glottolog}
\caption{Glottolog 3.2 for the doculects we used}
\end{figure}

\subsubsection{North Sea Germanic vs. Other}
\citet{stiles2013pan-west} posits that the most significant division of West Germanic varieties is the split into Ingv\ae{}onic (that is, North Sea Germanic) varieties and non-Ingv\ae{}onic varieties.

Ingv\ae{}onic: Frisian, English, and ``to a certain extend, Old Saxon''.

The distinct properties of the Ingv\ae{}onic subgroup
concern mostly inflection and pronouns (\citet{stiles2013pan-west}; \citet[pp. 7-8]{harbert2007germanic}),
although \citet{stiles2013pan-west} also lists some phonological characteristics:
``backing of long and short *\textit{a} before nasals[...]; fronting of long and short *\textit{a}; and palatalization of velar consonants''

see also the map based on harbert 2007



\subsubsection{The High German sound shift}

\citep[p. 47]{harbert2007germanic}
The voiceless (aspirated) Germanic stops (/*p, *t, *k/)
underwent lenition and shifted into affricates or fricatives.
Generally, these stops developed into fricatives in postvocal positions,
and into affricates in word-inital or postconsonantal positions,
and /*t/ changed more commonly than /p*/, which changed more commonly than /*k/.
South has more changes than North, muddy middle areas, see also Goblirsch's discussion

\citep[p. 48]{harbert2007germanic}
The voiced Germanic stops (/*b, *d, *g/) on the other hand
developed into their voiceless (and aspirated) counterparts.

\citeauthor{goblirsch2005lautverschiebungen} introduces
two related ways of categorizing German(ic) data based on obstruent features:
a four-way grouping of German dialects
based on in how far they have realized the High German sound shift \citeyearpar[pp. 182-199]{goblirsch2005lautverschiebungen},
and a five-way categorization of Germanic dialects
based on their obstruent features \citeyearpar[pp. TODO]{goblirsch2005lautverschiebungen}.

based on these two categorizations:

\begin{itemize}
  \item pf, ts, kx
  \item pf, ts
  \item ts
  \item --
\end{itemize}

% \begin{itemize}
%   \item HG consonant shift: voiceless plosives to affricates 
%   \begin{itemize}
%     \item Suedoberdeutsch (Suedbairisch, Suedmittelbairisch, Suedalemannisch, parts of Mittelalemannisch): pf, ts, kx
%     Ortisei
%     \item Nordoberdeutsch (Nordbairisch, Mittelbairisch, Nordalemannisch, parts of Mittelalemannisch, Ostfraenkisch, Suedrheinfraenkisch).
%     similar to Std German wrt the HGCS
%     p, t - pf, ts; kV - aspirated
%     \item Suedmitteldeutsch (Moselfraenkisch, Rheinfraenkisch, parts of Thueringisch, parts of Obersaechsisch, Schlesisch)
%     \item Nordmitteldeutsch (Ripuarian, Nordhessisch, Nordthueringisch, Nordobersaechsisch):
%     Vp, Vt, Vk - Vf, Vs, Vx; t - ts
%     Cologne
%   \end{itemize}
% \end{itemize}
% the ts, pf ts, pf ts kx distinction also appears in Harbert p. 48

% Biel, Hard, Graubuenden, Tuebingen, Walser - N/S Alemannic?

% Herrlisheim?

% Luxembourg?

% not affected by HGCS: Dutch, Low German, Frisian

% \subsubsection{Germanic varieties by obstruent systems}
% \citet[pp. 215-235]{goblirsch2005lautverschiebungen}: "Typology of modern Germanic obstruent systems", from least to most innovative

% considers whether obstruents are plosives (vs. affricates, fricatives), 

% \begin{itemize}
%   \item aspirationless:
%   (continental) Dutch, S Low German (Südwestfälisch, Südostfälisch), W Frisian, certain Central German dialects (incl. Ripuarian)
%   Dutch: Dutch Std NL, Dutch Std BE, Ostend, Antwerp, Limburg
%   S Low German: ???
%   W Frisian: Grou
%   Ripuarian: Cologne--but Cologne does have aspiration in our data
%   two series of plosives distinguished by voicing; one series of fricatives (217)
%   North Germanic: gemination
%   no (consistent) sound shift
%   occlusion, voice, length
%   \item general:
%   Low German, NE Dutch
%   partial sound shift
%   occlusion, voice, length (NG), aspiration
%   WG: simplified the geminates
%   CWG: final devoicing
%   aspiration
%   \item voiceless:
%   North Frisian; Suedmitteldeutsche Dialekte ohne Lenisierung
%   complete 2nd sc, but no lenition
%   occlusion, aspiration, length (NG)
%   \item length:
%   Oberdeutsche Dialekte ohne Lenisierung
%   \item aspiration:
%   Danish; N Low German; High German dialects with lenition
% \end{itemize}


\newpage
\section{Data}

continental European West Germanic dialects (CWG) and standard languages

\citet{heggarty2018sound} compiled IPA transcriptions of word lists in a range of Germanic doculects for his Sound Comparisons project.
From this database, we used 110 cognate sets from 20 modern CWG doculects and a reconstruction of Proto-Germanic.
For the phonetic alignment step (see section~\ref{s:alignments}), we used 14 additional doculects that are Germanic but not continental West Germanic. 
To control for transciber bias, we only worked with doculects that shared the same transcriptor, Warren Maguire.
We excluded one CWG doculect that covered only 35 concepts. % Schaeddel (Frisian)
The Proto-Germanic data cover all 110 concepts; each of the modern doculects covers at least 103 concepts, and each concept is covered by at least 17 modern doculects.
In total, we have 2181 sequence alignments between Proto-Germanic and modern CWG doculects.

note on inflected forms

\newpage
\section{Methods}

\subsection{Multiple Sequence Alignment}
\label{s:alignments}

In order to extract sound correspondences, we need to align the sound sequences corresponding to each concept first.
We carry out alignment based on data from all the investigated doculects at once using multiple sequence alignment.
Using multiple sequence alignment instead of carrying out pairwise alignments between the modern data and the reference data makes it possible to carry out the alignment based on patterns found in commonalities between the modern doculects as well.
Because of this, we use all of the Germanic data we extracted from the Sound Comparisons project instead of only the CWG doculects and Proto-Germanic.

We use a library-based version \citep{notredame2000t-coffee:} of the progressive multiple sequence alignment method \citep{thompson1994clustal}.
For each concept:

\begin{enumerate}
\item
Divide the phonetic representation of each word into an array of sound segments.
These sound segments are typically single IPA tokens (including diacritics), but we use multi-token segments for affricates, diph- and triphthongs, and geminates.

\item
Create alignments for all possible binary sequence combinations.
These alignments are created using the algorithm from \citet{needleman1970general},
with a scoring scheme based on the sound classes introduced by \citet{list2012sca}.
All segment alignments from this step are stored in a library,
each associated with a weight reflecting its relative frequency.
% TODO. in how far does it reflect probable sound class changes in the LingPy implementation (vs. relative frequency)

\item
Use the binary calculations to create a distance matrix between the sequences.
Convert the distance matrix into a tree using the UPGMA method \citep{sokal1958statistical}.

\item 
Progressing from the tips of the tree to the root,
consecutively join the alignments meeting at branchings,
until (at the root) all alignments have been consolidated into one alignment table.
Alignments are merged using the library created in the first step.
\end{enumerate}

We use the LingPy library for Python \citep{list2018lingpy} to perform these steps.
Table~\ref{tab:msa} shows an excerpt from the multiple sequence alignment for the concept ``cold''.

\begin{table}[h]
\begin{center}
\begin{tabular}{l|llllll}
\hline
Doculect       & \multicolumn{6}{l}{Sound segments} \\ \hline
Proto-Germanic  & k    & a    & l   & d    & a  & z  \\
Ortisei        & k\textsuperscript{h}   & \textopeno    & l   & \texttoptiebar{ts}  & -  & - \\ 
Biel           & \textchi    & \textscripta\textupsilon   & -   & t    & -  & -  \\
Walser         & x    & a\textlengthmark    & l   & t    & -  & -  \\
Luxembourg     & k\textsuperscript{h}   & a\textlengthmark   & l   & -    & -  & -  \\
Westerkwartier & k\textsuperscript{h}   & o    & \textltilde   & t\textsuperscript{h}   & -  & -  \\ \hline
\end{tabular}
\end{center}
\caption{An excerpt from the aligned sequence table for the concept ``cold''.}
\label{tab:msa}
\end{table}

\subsection{Sound Correspondence Extraction}

After performing sound segment-wise alignment,
we extract sound correspondences between each
modern doculect and Proto-Germanic from the alignment table.
We use straightforward segment-to-segment correspondences
as well as correspondences that include contextual information:

\begin{itemize}
\item
\textbf{No context}:
These are simple segment-to-segment correspondences.

\item
\textbf{Simple context}:
We (separately) add information about the
left and right single-segment context,
stating whether the context is a
word boundary, consonant, or vowel. 
This can only be performed when the context in question is of
the same type for both Proto-Germanic and the modern doculect.

\item
\textbf{Sound class-based context}:
This is similar to the previous category,
but we give more fine-grained information about consonants and vowels.
We use the sound classes introduced by \citet{list2012sca},
which discern between fifteen consonant groups and sixteen vowel groups.
We do not include information on word boundaries or gaps here,
as it would be identical to the word boundary- or gap-related information
included in the correspondences with simple contextual information.

\end{itemize}

% TODO check montemagni. also ref them here

Table~\ref{tab:corres} shows the sound correspondences
that can be inferred for the aligned segments
from Proto-Germanic and Ortisei German for the alignment
shown in Table~\ref{tab:msa}.

We ignore gap-gap alignments,
as they do not contain information on correspondences
between Proto-Germanic and the modern doculect in question,
only about an inserted sound segment in one or more other doculects.
Furthermore, we treat insertions and deletions
that LingPy flags as swaps (metathesis) as normal insertions/deletions,
as such cases only happen for 3 of the 111 concepts.

For each doculect, we ignore sound correspondences
that occurless than three times across all concepts
to reduce the effect misalignments can have. 

After extracting the sound correspondences for
all modern doculects, we have a doculect-by-correspondence
matrix storing the absolute frequencies of the sound correspondences.

\begin{table}[h]
\begin{center}
\begin{tabular}{l|llllll}
\hline
       & \multicolumn{6}{l}{Sound segments and inferred correspondences} \\ \hline

Proto-Germ.  & k    & a    & l   & d    & a  & z  \\
Ortisei        & k\textsuperscript{h}   & \textopeno    & l   & \texttoptiebar{ts}  & -  & - \rule[-2mm]{0pt}{0pt}\\\hline

No context & k $>$ k\textsuperscript{h} & a $>$ \textopeno & l $>$ l & d $>$ \texttoptiebar{ts} & a $>$ $\emptyset$ & z $>$ $\emptyset$ \rule{0pt}{4mm}\\[3mm]

Simple & k $>$ k\textsuperscript{h} / \#\_ & a $>$ \textopeno / cons\_ & l $>$ l / vow\_ & d $>$ \texttoptiebar{ts} / cons\_ & a $>$ $\emptyset$ / cons\_ & \\
context & k $>$ k\textsuperscript{h} / \_vow & a $>$ \textopeno{} / \_cons & l $>$ l / \_cons & & & z $>$ $\emptyset$ / \_\# \\[3mm]

Sound class- &  & a $>$ \textopeno / K\_ &  l $>$ l / A\_ & d $>$ \texttoptiebar{ts} / L\_ & & \\
based context & & a $>$ \textopeno{} / \_L & & & & \\
\hline
\end{tabular}
\end{center}
\caption{Proto-Germanic--Ortisei German sound correspondences extracted from the aligned entries for the concept ``cold''.}
\label{tab:corres}
\end{table}

\begin{table}[]
\begin{tabular}{llll}
Context type & Abbrev. & Exlanation & IPA characters\\\hline
\multirow{3}{*}{Simple context} & \# & word boundaries & \\
    & cons & consonants & \\
    & vow & vowels & \\[2mm]
\multirow{18}{*}{\begin{tabular}[c]{@{}l@{}}Sound class-\\ based context\end{tabular}}
    & A & unrounded back vowels          & a, \textscripta \\
    & B & labial/labiodental fricatives  & f, \texttoptiebar{pf}, v, \textphi, \textbeta\\
    & C & dental/alveolar affricates     & \texttoptiebar{dz}, \texttoptiebar{ts}, \texttoptiebar{t\textesh}\\ % not as context
    & D & dental fricatives              & \dh, \texttheta\\ % not as context
    & E & unrounded mid vowels           & e, \ae, \textturna, \textschwa, \textepsilon, \textrevepsilon, \textturnv \\
    & G & velar/uvual fricatives         & x, \textchi, \textgamma \\ % gamma not used as context
    & H & laryngeals                     & h, \texthth, \textglotstop \\ % only h used as context
    & I & unrounded close vowels         & i, \i \\
    & J & palatal approximants           & j \\
    & K & velar/uvular plosives          & k, \texttoptiebar{kx}, q, \textg \\
    & L & lateral approximants           & l, \textltilde, \textscl \\
    & M & labial nasals                  & m, \textltailm \\ % only m as context
    & N & (non-labial) nasals            & n, \ng, \textltailm, \textscn \\
    & O & rounded back vowels            & \textturnscripta\\ % not as context
    & P & labial plosives                & b, p \\
    & R & trills/taps/flaps              & r, \textturnr, \textfishhookr, \textscr, \textinvscr \\
    & S & sibilant fricatives            & s, z, \c{c}, \textesh, \textyogh, \textctj \\ % ctj not as context
    & T & dental/alveolar plosives       & t, d, \textrtailt \\
    & U & rounded mid vowels             & o, \o, \oe, \textopeno, \textbaro, \textscoelig \\ % scoelig not as context
    & W & labial approximants/fricatives & w \\
    & Y & rounded front vowels           & u, y, \textupsilon, \textscy\\\hline
\end{tabular}
\end{table}
% diacritics, diphthongs

\subsection{Clustering}

We implemented two approaches to custering the data.
Both clustering approaches follow a similar structure:
we first normalize the doculect-by-correspondence tally matrix
to adjust feature frequencies by how informative they are,
% dimensionality reduction,
then we perform hierarchical clustering.
Each approach is carried out once with only
the context-less sound correspondences,
and once with all context types.

\subsubsection{TF-IDF and UPGMA}

We first transform the frequencies
in the doculect-by-correspondence tally matrix
by applying TF-IDF weighting.

% TODO sources for tfidf?

Term frequency (TF) measures the relative frequency
of each sound correspondence within a doculect:

\begin{equation*}
\operatorname{tf}(doculect_i, corres_j) =
\frac{\text{number of occurrences of } corres_j \text{ in } doculect_i}
{\text{number of sound correspondences in } doculect_i}
\end{equation*}

while inverse document frequency (IDF)
considers how many doculects cover a given
sound correspondence:

\begin{equation*}
\operatorname{idf}(corres_j) =
log(
\frac{\text{number of doculects}}
{\text{number of doculects with } corres_j}
).
\end{equation*}

To combine term frequency and inverse document frequency
and transform the tally matrix, 
we use the implementation in the Python library scikit-learn
\citep{pedregosa2011scikit-learn},
where it is calculated as

\begin{equation*}
\operatorname{tf-idf}(doculect_i, corres_j) =
\text{tf}(doculect_i, corres_j)
\times
(
\text{idf}(corres_j)
+ 1).
\end{equation*}

% We then apply truncated singular vector decomposition (preserve a large enough number of components to explain at least 85\% of the variance)
% remove noise

We create a doculect-by-doculect distance matrix
with distances bounded between $0$ (identical) and $1$ (maximally different)
by calculating the cosine distances between each
binary combination of row vectors from the doculect-by-correspondence matrix:

\begin{equation*}
\operatorname{cosine\_distance}(doculect_i,doculect_j) =
1 -
\frac{doculect_i \cdot doculect_j}{\lVert doculect_i \rVert \lVert doculect_j \rVert}
.
\end{equation*}

We then convert this distance matrix into a dendrogram,
using the UPGMA method introduced by \citet{sokal1958statistical},
as UPGMA was found to be preferable to other
distance matrix-based hierarchical clustering methods
for analyzing dialect distances by \citet{heeringa2004measuring}.


\subsubsection{Bipartite Spectral Graph Co-clustering}

For the other clustering method, we use the approach
introduced by \citet{dhillon2001co-clustering}.
We follow \citeauthor*{wieling2009bipartite} who introduced
this method to dialectometry for flat clustering \citeyearpar{wieling2009bipartite} and hierarchical clustering \citeyearpar{wieling2010hierarchical}.

For this approach, we use a binary version of the
doculect-by-correspondence tally matrix that only
indicates whether a doculect exhibits
a sound correspondence ($1$) or not ($0$).

This method works as follows:

\begin{enumerate}
\item 
% Given a (binary) co-occurrence matrix $A^{m \times n}$ ($m$ = number of doculects, $n$ = number of sound correspondences), normalize this matrix.
We begin with normalizing the binary co-occurrence matrix
$A^{m \times n}$ ($m$ = number of doculects, $n$ = number of sound correspondences).
First, we create two diagonal matrices
$D_1^{m \times m}$ and $D_2^{n x n}$ that, respectively,
contain the row sums or column sums of $A$.
We use these diagonal matrices to reduce
the importance of doculects (or correspondences)
that co-occur with a large number
of sound correspondences (or doculects).
Thus we create the normalized matrix $A_n$
by dividing each entry in $A$ by
the square root of the sum of its row's entries
and by the square root of the sum of its column's entries:
\begin{equation*}
A_n = D_1^{-\frac{1}{2}} \times A \times D_2^{-\frac{1}{2}}.
\end{equation*}

\item
We perform singular value decomposition on $A_n$
to obtain the left and right singular vectors $u_i$ and $v_i$.
We ignore the singular vectors belonging
to the largest singular value as they do not contain
information relevant for partitioning the data \citep{kluger2003spectral},
and work with the second singular vectors ($u_2$, $v_2$) instead.
We calculate the vector $Z^{(m + n) \times 1}$ such that
% \begin{equation*}
% Z = 
% \begin{cases}
% D_1^{-\frac{1}{2}} \times u_2, & \text{for the first } m \text{ entries}\\
% D_2^{-\frac{1}{2}} \times v_2, & \text{for the last } n \text{ entries}
% \end{cases}
% \end{equation*}
\begin{align*}
Z_{[0, m]} = D_1^{-\frac{1}{2}} \times u_2\\
Z_{[m, m+n]} = D_2^{-\frac{1}{2}} \times v_2
.
\end{align*}
% TODO make this understandable

% We calculate $D_1^{-\frac{1}{2}} \times u_2$
% and $D_2^{-\frac{1}{2}} \times v_2$,
% and concatenate them to the vector $Z$. 

\item
We perform k-means clustering on $Z$ with $k = 2$.

\item
For each cluster that contains at least two doculects,
we create the binary co-occurrence matrix $A$ describing
the doculects and sound correspondences in this cluster,
and repeat steps 1--3.
\end{enumerate}

% TODO sources for svd, kmeans nec?

If a cluster produced in step 3 contains
sound correspondences that are not exhibited
by any of the doculects in this cluster,
we assign this correspondence to the other cluster.
This only happens rarely, and in these cases the corresponding
value in $Z$ is near the k-means decision boundary.
We need to change the cluster identity in such cases,
as it would otherwise not be possible to normalize
the cluster's co-occurrence matrix when partitioning
the cluster elements again.
% TODO do parts of this belong in the results section??

\subsection{Ranking sound correspondences by importance}

When all doculects have been assigned to this hierarchical cluster structure,
we rank the sound correspondences associated with each cluster.
In the case of the TF-IDF method, these are
the sound correspondences exhibited by each cluster's doculects;
for the graph clustering method,
these are the sound correspondences that are in the same cluster.

We use the representativeness and distinctiveness metrics
introduced by \citet{wieling2011bipartite},
as well as a modified version of their importance measure.

Representativeness measures how many doculects in a given cluster
exhibit a given sound correspondence:

\begin{equation*}
\operatorname{rep}(cluster_i, corres_j) = 
\frac{\text{number of doculects in } cluster_i \text{ with }  corres_j}
{\text{number of doculects in }  cluster_i}
.
\end{equation*}

Representativeness is bounded between
$0$ (no doculects in the cluster show the given sound correspondence)
and $1$ (all doculects in the cluster do).

Distinctiveness indicates how often a given sound correspondence
occurs in a given cluster compared to other clusters. 
This requires two additional measures:
relative occurrence,hich indicates the proportion
of doculects exhibiting the sound correspondence
that is in the given clusters,
and relative size, which gives the number of doculects 
in the cluster relative to the number of all examined modern doculects:

\begin{align*}
\operatorname{relative\_occurrence}(cluster_i, corres_j) &= 
\frac{\text{number of doculects in } cluster_i \text{ with }  corres_j}
{\text{total number of doculects with } corres_j}\\
\operatorname{relative\_size}(cluster_i) &= 
\frac{\text{number of doculects in } cluster_i}
{\text{total number of doculects}}
.
\end{align*}

These two concepts are combined to determine the distinctiveness score:

\begin{equation*}
\operatorname{dist}(cluster_i, corres_j) = 
\frac{\operatorname{relative\_occurrence}(cluster_i, corres_j) - \operatorname{relative\_size}(cluster_i)}
{1 - \operatorname{relative\_size}(cluster_i)}
.
\end{equation*}

Distinctiveness has an upper bound of
$1$ (the sound correspondence only occurs in the given cluster),
but no lower bound.
A value of $0$ indicates that the sound correspondence
has the same relative frequency within the cluster
as among the total set of doculects.
Negative values indicate that the sound correspondence
has a lower relative frequency within the cluster
than among all doculects.

Importance is the average of representativeness and distinctiveness.
\citet{wieling2011bipartite} use the arithmetic mean
and mention the possibility of exploring
other ways of combining the two metrics.
We use the harmonic mean in order to penalize cases
where the representativeness value is very high
but the distinctiveness value is very low (or vice versa).
We also assign an importance score of $0$ to
cases with negative distinctiveness values:

\begin{equation*}
\operatorname{imp}(cluster_i, corres_j) = 
\begin{dcases}
\frac{
2 * \operatorname{rep}(cluster_i, corres_j) * \operatorname{dist}(cluster_i, corres_j)}
{\operatorname{rep}(cluster_i, corres_j) + \operatorname{dist}(cluster_i, corres_j)}, & \text{if dist}(cluster_i, corres_j) > 0\\
0, & \text{otherwise}.
\end{dcases}
\end{equation*}

We additionally re-rank correspondences
with the same importance score
such that more frequent correspondences rank higher.

\newpage
\section{Results}

Applying the aforementioned methods yields
four arrangements of the data into hierarchical partitions,
two for the TFIDF method and two for the graph clustering method.

\subsection{TFIDF}

Figure~\ref{fig:tfidf-dendrograms} shows the dendrograms
created by the TFIDF method for sound correspondences
including and excluding contextual information.
Of the 18 intermediary clusters
(i.e. clusters that are neither singletons nor contain all doculects),
13 are associated with a sound correspondence with an importance score of at least 70\% for the context-less run,
and 17 for the run with additional contextual information. 

In total, 201 sound correspondence were used for the run without contextual information,
of which 6 have importance scores of 100\% for intermediary clusters.
For the run with contextual information,
24 sound correspondences (of 665 total) reach 100\% importance for intermediary clusters.
Tables~\ref{tab:tfidf-nocontext-corres} and \ref{tab:tfidf-context-corres}
show the highest-rank sound correspondences (importance score $\geq$ 90\%)
for the context-less and context information run, respectively.

\begin{figure}
    % \centering
    % \begin{adjustwidth}{-0.6cm}{-0.5cm}
    % height=0.28\textheight
    \includestandalone[height=0.4\textheight]{figures/tfidf-nocontext}
    % \hspace{-5mm}
    \includestandalone[height=0.4\textheight]{figures/tfidf-context}
    % \end{adjustwidth}
    \caption{TFIDF with no (top) and additional (bottom) context information, highest-rating correspondence per non-singleton cluster (with $\geq$70\% importance score)}
    \label{fig:tfidf-dendrograms}
\end{figure}

\begin{table}[h]
\begin{tabular}{p{7cm}p{2.5cm}rrrrc}
\hline
Cluster & Sound corres. & Imp & Rep & Dist & Count \\ \hline

Cologne, Luxembourg & x $>$ \textesh & 100 & 100 & 100 & 6 \\
    & x $>$ \textesh / vow\_ & 100 & 100 & 100 & 6 \\ [2mm]

Graubuenden, Walser & a $>$ \textsubbar{a}\texthalflength{} / cons\_ & 100 & 100 & 100 & 7\\[2mm]

Antwerp, Ostend & x $>$ $\emptyset$ / \_vow & 100 & 100 & 100 & 17\\
    & r $>$ \textsubbar{s} & 100 & 100 & 100 & 11\\
    & r $>$ \textsubbar{s} / vow\_ & 100 & 100 & 100 & 10\\
    & k $>$ \textsubplus{k} / vow\_ & 100 & 100 & 100 & 6\\[2mm]

Herrlisheim, Ortisei & a $>$ \textopeno & 100 & 100 & 100 & 10 \\
    & a $>$ \textopeno / \_cons & 100 & 100 & 100 & 10 \\
    & a $>$ \textopeno / cons\_ & 100 & 100 & 100 & 9 \\
    & r $>$ \textchi / \_cons & 100 & 100 & 100 & 9 \\[2mm]

Antwerp, Dutch Std BE, Ostend & f $>$ v & 100 & 100 & 100 & 15\\
    & f $>$ v / \_vow & 100 & 100 & 100 & 13\\
    & k $>$ \textsubplus{k} & 100 & 100 & 100 & 13\\
    & f $>$ v / \#\_ & 100 & 100 & 100 & 12\\[2mm]

Feer, Heligoland & d $>$ $\emptyset$ / cons\_ & 100 & 100 & 100 & 10\\
    & d $>$ $\emptyset$ / N\_ & 100 & 100 & 100 & 8\\
    & s $>$ s / \_K & 100 & 100 & 100 & 8\\[2mm]

Tuebingen, Cologne, Luxembourg, & $\emptyset$ $>$ \textglotstop{} / \#\_ & 100 & 100 & 100 & 169\\
Westerkwartier, Feer, Heligoland, Biel, & $\emptyset$ $>$ \textglotstop{} / \_vow & 100 & 100 & 100 & 169\\
Graubuenden, Walser, Hard, & $\emptyset$ $>$ \textglotstop{} & 100 & 100 & 100 & 169\\
Herrlisheim, Ortisei & t $>$ \texttoptiebar{ts} & 91 & 83 & 100 & 76\\
    & k $>$ k\textsuperscript{h} & 91 & 83 & 100 & 52\\
    & t $>$ \texttoptiebar{ts} / \#\_ & 91 & 83 & 100 & 49\\
    & k $>$ k\textsuperscript{h} / \#\_ & 91 & 83 & 100 & 46\\
    & t $>$ \texttoptiebar{ts} / \_vow & 91 & 83 & 100 & 40\\
    & k $>$ k\textsuperscript{h} / \_vow & 91 & 83 & 100 & 38\\
    & r $>$ $\emptyset$ / \_vow & 91 & 83 & 100 & 30\\
    & ll $>$ l & 91 & 83 & 100 & 30\\
    & ll $>$ l / vow\_& 91 & 83 & 100 & 30\\[2mm]

Grou, Dutch Std NL, Dutch Std BE, & k $>$ k / \#\_ & 100 & 100 & 100 & 40\\
Antwerp, Ostend, Limburg, & t $>$ t / \_\# & 100 & 100 & 100 & 37\\
Achterhoek, Veenkolonien & k $>$ k / \_vow & 100 & 100 & 100 & 31\\\hline
\end{tabular}
\caption{TFIDF: sound correspondences with an importance score of 90\% or higher.
Importance, representativeness, and distinctiveness scores are percentages and rounded to the nearest integer.
context
}
\label{tab:tfidf-context-corres}
\end{table}

\begin{table}[h]
\begin{tabular}{llrrrrc}
\hline
Cluster & Sound corres. & Imp & Rep & Dist & Count\\ \hline

Cologne, Luxembourg & x $>$ \textesh & 100 & 100 & 100 & 6\\[2mm]

Herrlisheim, Ortisei & a $>$ \textopeno & 100 & 100 & 100 & 10\\[2mm]

Heligoland, Westerkwartier & t $>$ \textsubring{d} & 100 & 100 & 100 & 8\\[2mm]

Antwerp, Dutch Std BE, Ostend & f $>$ v & 100 & 100 & 100 & 15\\
& k $>$ \textsubplus{k} & 100 & 100 & 100 & 13\\[2mm]

Biel, Graubuenden, Hard, Walser & e $>$ y\textlengthmark & 100 & 100 & 100 & 13\\[2mm]

Herrlisheim, Ortisei, Hard, Biel, Graubuenden, & t $>$ s & 90 & 82 & 100 & 59\\
Walser, Limburg, Feer, Tuebingen, Cologne, Luxembourg & & & & & \\[2mm]

Heligoland, Westerkwartier, Herrlisheim, Ortisei, Hard, & $\emptyset$ $>$ \textglotstop{} & 96 & 92 & 100 & 169\\
Biel, Graubuenden, Walser, Limburg, Feer, Tuebingen, & s $>$ \textesh & 92 & 85 & 100 & 107\\
Cologne, Luxembourg & & & & & \\\hline
\end{tabular}
\caption{TFIDF: sound correspondences with an importance score of 90\% or higher.
Importance, representativeness, and distinctiveness scores are percentages and rounded to the nearest integer.
nocontext}
\label{tab:tfidf-nocontext-corres}
\end{table}



\subsection{BSGC}

\begin{figure}
  \centering
  \includestandalone[height=0.35\textheight]{figures/bsgc-context}
  
  \vspace{2em}
  
  \includestandalone[height=0.35\textheight]{figures/bsgc-nocontext}
  \caption{BSGC with (top) and without (bottom) context information and highest-rating correspondence per non-singleton cluster (with $\geq$70\% importance score)}
  \label{fig:bsgc-trees}
\end{figure}

% CONTEXT
% excl. singletons and the cluster including all doculects
% max. importance score: 100.00%
% 665 correspondences (total)
% 63 correspondences >= the threshold (70% importance)
% 9 correspondences >= 90% importance
% 3 correspondences >= 95% importance
% 2 correspondences == 100% importance

% NO CONTEXT
% excl. singletons and the cluster including all doculects
% max. importance score: 100.00%
% 201 correspondences (total)
% 22 correspondences >= the threshold (70% importance)
% 1 correspondences >= 90% importance
% 1 correspondences >= 95% importance
% 1 correspondences == 100% importance


\begin{table}[h]
\begin{tabular}{llrrrrc}
\hline
Cluster & Sound corres. & Imp & Rep & Dist & Count\\ \hline

Heligoland, Westerkwartier & t $>$ t\textsuperscript{h} / vow\_ & 100 & 100 & 100 & 14\\[2mm]

Antwerp, Dutch Std BE & e $>$ e\textlengthmark & 100 & 100 & 100 & 7\\[2mm]

Biel, Cologne, Feer, Graubuenden, Hard, Herrlisheim, & t $>$ s & 90 & 82 & 100 & 59\\
Limburg, Luxembourg, Ortisei, Tuebingen, Walser &  t $>$ s / vow\_ & 90 & 82 & 100 & 58\\
&  t $>$ s / \_\# & 90 & 82 & 100 & 34\\[2mm]

Biel, Cologne, Feer, Graubuenden, Hard, Heligoland, Herrlisheim, & $\emptyset$ $>$ \textglotstop{} / \_vow & 96 & 92 & 100 & 169\\
Limburg, Luxembourg, Ortisei, 
Tuebingen, Walser, Westerkwartier & s $>$ \textesh & 92 & 85 & 100 & 107\\
 & s $>$ \textesh / \#\_ & 92 & 85 & 100 & 63\\
 & s $>$ \textesh / \_cons & 92 & 85 & 100 & 60\\\hline
\end{tabular}
\caption{BSGC: sound correspondences with an importance score of 90\% or higher.
Importance, representativeness, and distinctiveness scores are percentages and rounded to the nearest integer.
context}
\label{tab:bsgc-context-corres}
\end{table}

\begin{table}[h]
\begin{tabular}{llrrrrc}
\hline
Cluster & Sound corres. & Imp & Rep & Dist & Count\\ \hline

Antwerp, Dutch Std BE & e $>$ e\textlengthmark & 100 & 100 & 100 & 7\\\hline
\end{tabular}
\caption{BSGC: sound correspondences with an importance score of 90\% or higher.
Importance, representativeness, and distinctiveness scores are percentages and rounded to the nearest integer.
nocontext}
\label{tab:bsgc-nocontext-corres}
\end{table}

% svd how many components. svd vs no svd

high representativeness score for all doculects?


\newpage
\section{Discussion}

% -- specific ---

\subsection{Comparisons to proposed groupings/structures}
\subsubsection{Comparison with Glottolog tree}
comparison with language ancestry data from \citet{hammarstrom2018glottolog}

\subsubsection{Ingv\ae{}onic vs. Non-Ingv\ae{}onic}
gold standard based on \citet{stiles2013pan-west}:
\begin{itemize}
\item
Ingv\ae{}onic:
Achterhoek, Feer, Grou, Heligoland, Veenkolonien, Westerkwartier
\item
Non-Ingv\ae{}onic:
the rest
\end{itemize}

\subsubsection{obstruent systems}

\subsubsection{(grouping dialects by) sound shifts}

do the groups/criteria match the ones introduced earlier?

\subsection{effect of context information}

really helps for BSGC: higher importance values! (more values $\geq$80, scores $\geq$90 at all)

\subsection{co-clustering}
How beneficial is co-clustering really? Why not just cluster the doculects (after TF-IDF maybe) and then apply the sound correspondence metrics? It seems like the benefit of co-clustering is mainly that $v_2$ can be used for ranking the correspondences, although this doesn't seem to be very popular among the authors who used bipartite spectral graph clustering for dialect data...

why does bsgc not work that well here despite it working quite nicely for wieling, montemagni, ...?
- number of samples
- doculects/language family ("tighter")
- narrowness of transcription

wrongly assigned sound correspondences (k means initialization?)

comparison to the results discussed in \citet{wieling2011bipartite}
They remark on a common alignment [-]:[\textesh], which commonly appears after [t]:[t]. Interpreting affricates as single segments with the result of correspondences such as [t]:[\t{t\textesh}], or using another approach to include contextual information seems more satisfying to me.

% --- more general ---

All of the comparisons here are phonetic (and might possibly include some morphological information in some cases) and on a word level, but we are ignoring lexical, syntactical, morphological, etc. information in this analysis.

To what extend does clustering dialects and/or applying a hierarchical model to dialect data make sense? (tree vs. web, `vertical' changes vs. `horizontal' influences)
data is from a relatively large region though

more abstract sound correspondence rules (incl. context) to model sound changes more generally (nasalization, lenition)

Regularity: How often does this correspondence occur compared to the original sound? similar to \citet{prokic2013combining}. also like bigram MLE

investigate further the influence of data selection/preprocessing/properties on bsgc performance

\newpage
\bibliographystyle{chicago}
\bibliography{lib}
\end{document}

% NOTES
% intros to SVD, k-means, tf-idf, alignments?
